\section{\textcolor{\getcol{\thesection}}{Auszeichnungen}}
\ifextended\vspace{-0.8\baselineskip}\else\fi

\begin{center}
	\ff
	\renewcommand{\arraystretch}{1.7}
	\renewcommand{\cellalign}{lt}
	\begin{tabularx}{\textwidth}{ @{} l @{}p{\dist}@{} X @{}}
		\tym{2020}{09} && \job{Herausragende Gesamtarbeit im P2 Versuch VIR} \hfill \coy{LMU München}{DE} \\[-0.7em]
		&& \smaller{1}{Verliehen für eine besonders herausragende Lösung im Versuch "Viskosität und Reynoldszahl" (VIR), die weit über die Mindestanforderungen hinausgeht.} \\
		\tym{2018}{11} & & \textbf{Empfänger der Kommandantenmünze} \hfill \coy{Streitkräfte Singapurs}{SG} \\[-0.7em]
			&& \smaller{1}{für herausragende Arbeit} \\
		\tym{2016}{06} && \job{Fachpreis: H2 Deutsch} \hfill \coy{Hwa Chong Institution}{SG} \\[-0.7em]
			&& \smaller{1}{Bester im Fach im gesamten Jahrgang} \\ % für das Vorjahr
		\tym{2016}{06} && \job{Hwa Chong Diploma mit Auszeichnung} \hfill \coy{Hwa Chong Institution}{SG} \\[-0.7em]
			&& \smaller{1}{Top 1\% des nationalen Jahrgang}\\
		\tym{2016}{03} && \job{National Olympiad for Informatics (NOI), Bronzemedaille}  \\[-0.7em]
			&& \hfill\coy{National University of Singapore}{SG}  \ifextended \\[-1em] \else\fi
	\end{tabularx}
\end{center}