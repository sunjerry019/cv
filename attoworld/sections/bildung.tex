\section{\textcolor{\getcol{\thesection}}{Bildung}}
\vspace{-\baselineskip}

\begin{center}
	\ff
	\renewcommand{\arraystretch}{1.9}
	\renewcommand{\cellalign}{lt}
	\begin{tabularx}{\textwidth}{ @{} l @{}p{\dist}@{} X @{}}
		\tym{2019}{10} - \texttt{Jetzt}
			&& \job{Ludwig-Maximilians-Universität München}, DE \hfill \textcolor{schtitles}{\smaller{1}{\textit{Universität}}} \\[-1em]
			&& Physik, B.Sc. \hfill 4. Semester \\[-0.5em]
			&& \smaller{1}{Neben dem Physik-Curriculum eigenständiges Studium der Informatik mit Lehrveranstaltungen der LMU und der TU München}  \\
% 			Interdisziplinäre Ausbildung in der Physik und Informatik, auch mit Vorlesungen von dem Institut für Informatik an der TU München.
		\midrule
		\tym{2016}{12}
			&& \job{School Graduation Certification}, SG \hfill \textcolor{schtitles}{\smaller{1}{\textit{Schulabschluss}}} \\[-1em]
			&& \smaller{1}{Singapore-Cambridge GCE Advanced Level Examination \hfill [Note: $1,1$]} \\
		\midrule

		\tym{2015}{01} - \tym{2016}{12}
		&& \job{Hwa Chong Institution (College)}, SG \hfill \textcolor{schtitles}{\smaller{1}{\textit{voruniversitäre Bildung}}} \\[-1em]
			&& \smaller{2}{--- \textit{Gifted And Talented Education Programme} (GATE) in Physik und Mathematik} \\[-1em]
			&& \smaller{2}{\phantom{---} \hspace{1em} Top $10$\% des Jahrgangs} \\[-1em]
			&& \smaller{2}{--- \textit{Centre for Science Research \& Talent Development Programme} (CenTaD)} \\

		\tym{2011}{01} - \tym{2014}{12} 
		&& \job{Hwa Chong Institution (High School)}, SG \hfill \textcolor{schtitles}{\smaller{1}{\textit{Sekundarschule}}} \\[-1em]
			&& \smaller{2}{--- \textit{Special Science and Math Talent Programme} (SSMT) \hfill \highlight{Top $5$\% des Jahrgangs}} \\[-1em]
			&& \smaller{2}{--- [\texttt{2014}] \highlight{Forschungsaustausch} mit der \textit{Loudoun Academy of Science} \hfill \country{Virginia, US}} \\[-1em]
			&& \smaller{2}{\phantom{---} Gemeinsame Arbeit an einem Forschungsprojekt zum Thema Bioremediation} \\
	\end{tabularx}
\end{center}