\section{\textcolor{\getcol{\thesection}}{Relevante Programmiererfahrung}}
% \vspace{-\baselineskip}
{
	% \renewcommand{\arraystretch}{2}
	\renewcommand{\cellalign}{lc}
	% \setlength{\tabcolsep}{0.3cm}
	\begin{tabularx}{\textwidth}{@{} X p{0.20cm} r @{}}
		\textbf{photonLauncher} \hfill \badge{python/bash} & & \texttt{2019} \\
		\urllinkout{https://github.com/sunjerry019/photonLauncher}{github.com/sunjerry019/photonLauncher} & & \texttt{\hfill \hspace{0.1mm} $\wedge$ \hfill} \\
		\hspace{1em} \smaller{2}{\color{gray} Als Teil der Forschungsprojekt:} & & \texttt{2014} 
		% \smaller{1}{Created equipment interfacing scripts used on Linux with \code{pyusb} and automated said scripts for repeated data acquisition. \code{numpy} and \code{gnuplot} were used to statiscally analyse collected data. Reverse \code{SSH} was utilized for remote monitoring, experimentation and extraction of data without physical access to the lab.} & & \\
	\end{tabularx}
	\blanko\hspace{2em} \smaller{2}{\color{gray} Poster "Measuring the Temporal Correlation of Light from a Mercury Discharge Lamp" \hfill \texttt{2016}}
	\begin{itemize}[label={--}]
		\item \smaller{2}{Entwicklung von Skripten für die Geräteschnittstelle mittels \texttt{pyusb}, um die Automatisierung von Messungen zu ermöglichen.} \vspace{-0.5em} % Unter anderem wurden Geräte von ThorLabs und Micos verwendet
		\item \smaller{2}{Analyse von Daten mittels \code{numpy} und \code{gnuplot}} \vspace{-0.5em}
		\item \smaller{2}{Aufbau der Labornetzwerkinfrastruktur, u.a. Implementierung der \textit{reverse SSH}, um Steuerung von Geräten von außerhalb des Labors zu ermöglichen.} \vspace{-0.6em}
	\end{itemize}
}
\vspace{1.5\parskip}
\begin{tabularx}{\textwidth}{@{} l X @{}}
	\textbf{Außerdem Erfahrung mit:} & \code{HTML/CSS/JS}, \code{PHP}, \code{C++}, \code{node.js}, \code{MATLAB}, \hfill \textit{\smaller{1}{\linkout{https://github.com/sunjerry019}{[GitHub]}}} \\
	& \smaller{1}{Linux Systemverwaltung}
\end{tabularx}
% \vspace{-2\baselineskip}
\newpage