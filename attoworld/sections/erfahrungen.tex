\section{\textcolor{\getcol{\thesection}}{Praktische Erfahrung}} %Fachbezogene Arbeitserfahrung
\vspace{-\baselineskip}

\begin{center}
	\ff
	\renewcommand{\arraystretch}{1.9}
	\renewcommand{\cellalign}{lt}
	\begin{tabularx}{\textwidth}{ @{} l @{}p{\dist}@{} X @{}}
		\tym{2020}{10} - Jetzt
			&& \job{Tutor beim Physikpraktikum für Humanmediziner} \hfill \coy{LMU München}{DE} \\[-0.7em]
			&& \smaller{1}{Betreuung der Versuche "Akustische und elektrische Signale" und "Linsen"} \\
		\tym{2019}{02} - \tym{2019}{09}
			&& \job{Forschungspraktikant} \hfill \coy{National University of Singapore}{SG} \\[-1em]
			&& \smaller{1}{Nanomaterials Research Lab} \\[-0.5em]
			&& \smaller{2}{--- Laser-Assisted Modifikation von Pflanzenoberflächen auf mikroskopischer Ebene} \\[-1em]
			&& \smaller{2}{--- Automatisierung von Datensammlungs- und Datenverarbeitungsprozessen} \\[-1em]
			&& \smaller{2}{--- Vorführungen, Workshops und Laborrundgänge für Schüler} \\
% 			Wissenschaftskommunikation
			\tym{2019}{02} - \tym{2019}{09}
			&& \job{Tätigkeit als Selbstständiger Softwareentwickler} \hfill \coy{}{SG}\\[-0.7em]
			&& \smaller{2}{--- Full-Stack Webentwicklung und Verwaltung von Datenbanken} \\[-1em]
			&& \smaller{2}{--- Automatisierung von Prozessen mittels Python} \\
		\tym{2017}{02} - \tym{2019}{02}
			&& \job{Verpflichtender Militärdienst} \hfill \coy{Streitkräfte Singapurs}{SG} \\
		\tym{2017}{01} - \tym{2017}{02}
			&& \job{Lehrpraktikant} \hfill \coy{Queensway Secondary School}{SG} \\[-0.7em]
			&& \smaller{1}{Unterrichten von 9. und 10. Klassen in Physik und Mathematik} \\
		\tym{2014}{01} - \tym{2016}{06}
			&& \job{Gründungsmitglied der Photonik AG} \\[-1em]
			&& \hfill \coy{Hwa Chong Science Research Center, Photoniklabor}{SG} \\[-0.5em]
			&& \smaller{2}{--- Aufbau des Labors, Programmierung von Software für Geräteschnittstellen} \\[-1em] % Erstellung der Skripte
			&& \smaller{2}{--- Statistiche Korrelationsanalyse von Photon Bunching mit Silizium Avalanche-Pho\-to\-nen\-detektoren} \\[-0.7em]
			&& \smaller{2}{\texttt{2016} \hspace{0.5em} Poster-Präsentation beim \textit{Institute of Physics Singapore} (IPS) \textit{Meeting}:} \\[-1em]
			&& \smaller{2}{\phantom{\texttt{2016}} \hspace{0.5em} {\paper{"Measuring Temporal Coherence of Light from a Mercury Vapour Lamp"}}} \\
% 			&& \smaller{2}{\texttt{2016} \hspace{0.5em} \textit{Measuring Temporal Coherence of Light from a Mercury Vapour Lamp}} \\[-1em]
% 			&& \smaller{2}{\texttt{2016} \hspace{0.5em} Institute of Physics Singapore (IPS) Meeting, Poster Beiträger}
		\tym{2011}{01} - \tym{2016}{06}
			&& \job{Mitglied der Computer AG} \hfill \coy{Hwa Chong Institution}{SG} \\[-0.7em]
			&& \smaller{1}{Stellvertretender Vorsitzender im Jahr \texttt{2014}} 
	\end{tabularx}
\end{center}


% \begin{tabularx}{0.9\textwidth}{  p{4cm}  X  }
		
% \makecell{\texttt{\footnotesize von} \hspace{2.4em} \texttt{\footnotesize bis} \\ \texttt{2019{\footnotesize /02}} - \texttt{2019{\footnotesize /09}}} & \makecell{\job{Forschungspraktikant} \\ 
% 	{\small National University of Singapore, SG} \\ 
% 	{\scriptsize Nanomaterials Research Lab} 
% 	\\ {\scriptsize Dazu auch Betreuung von Schüler aus \country{China}, \country{Taiwan}, \country{Indien}, u.a.}
% 	\\ --- \texttt{2019{\footnotesize /07}} {\scriptsize Wissenschaft Sommercamp}
% 	\\ --- \texttt{2019{\footnotesize /06}} {\scriptsize Physik-Anreichungscamp} 
% } \\
% \makecell{\texttt{\footnotesize von} \hspace{2.4em} \texttt{\footnotesize bis} \\ \texttt{2017{\footnotesize /02}} - \texttt{2019{\footnotesize /02}}} & \makecell{\job{Militärdienst}\\{\small Streitkräfte Singapurs, SG}} \\
% \makecell{\texttt{\footnotesize von} \hspace{2.4em} \texttt{\footnotesize bis} \\ \texttt{2017{\footnotesize /01}} - \texttt{2017{\footnotesize /02}}} & \makecell{\job{Lehrpraktikant}\\{\small Queensway Secondary School, SG}}